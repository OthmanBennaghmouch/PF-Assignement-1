\documentclass[12pt]{article}
 \usepackage[margin=1in]{geometry} 
\usepackage{amsmath,amsthm,amssymb,amsfonts}
 
\newcommand{\N}{\mathbb{N}}
\newcommand{\Z}{\mathbb{Z}}
 
\newenvironment{problem}[2][Exercise]{\begin{trivlist}
\item[\hskip \labelsep {\bfseries #1}\hskip \labelsep {\bfseries #2.}]}{\end{trivlist}}
%If you want to title your bold things something different just make another thing exactly like this but replace "problem" with the name of the thing you want, like theorem or lemma or whatever
 
\begin{document}
 
%\renewcommand{\qedsymbol}{\filledbox}
%Good resources for looking up how to do stuff:
%Binary operators: http://www.access2science.com/latex/Binary.html
%General help: http://en.wikibooks.org/wiki/LaTeX/Mathematics
%Or just google stuff
 
\title{Principle of Finance : Assignment 1}
\author{Othman Bennaghmouch \and Arnaud Rubin \and David Maurissen}
\date{October 7, 2018}
\maketitle
 
\begin{problem}{1}

\end{problem}
 
For PV the present value, $C$ the cash flow and r the interest rate, we have $PV=\frac{C}{(1+r)^n}$ In this case, we have $PV=\frac{14,500}{1.075^2}+\frac{175,000}{1.075^5}$
 
 The value of the benefits of this project today is $PV=\$24,737.1$ .
 

\begin{problem}{2}

\end{problem}
 
 $NPV = PV(benefits)-PV(costs)$
 
 $NPV=\frac{5,000}{1.055}+\frac{5,000}{1.055^2}+\frac{5,000}{1.055^3}+\frac{5,000}{1.055^4}+\frac{5,000}{1.055^5}-(3,500+\frac{7,000}{1.055^2}+\frac{7,000}{1.055^3}+\frac{7,000}{1.055^4})$
 
 $NPV=-\$49.56$
 
 The NPV being negative, the investment shouldn't be undertaken.
 
 \begin{problem}{3}

\end{problem}
 
 $NPV = PV(benefits)-PV(costs)$
 
 $NPV=\frac{\$420,000}{1.03^3}-35,0000 CHF *1.09\frac{\$}{CHF}$
 
 $NPV=\$2,859.5$
 
 The NPV being positive, the investment should be undertaken.
 
 \begin{problem}{4}

\end{problem}
 
 From the table, we see clearly that P(A) + P(B) = P(C)\linebreak
 From the data of the exercise, we also know that CF(A) + CF(B)=80\$+100\$\linebreak
 
 As stated, CF(C)=180\$, and since we have CF(A)+CF(B)=CF(C), the law of one price applies and there is no arbitrage opportunity.
 
 
 
 \begin{problem}{5}

\end{problem}
 
 We have to calculate the NPV of both investments. \linebreak
 
 For investment A, we have $NPV=-100+\frac{45}{1.03}+\frac{75}{1.03^2}-\frac{10}{1.03^3}$ which give $NPV=5.23$
 For investment B, we have $NPV=-1,000+\frac{500}{1.03}+\frac{600}{1.03^2}-\frac{70}{1.03^3}$ which give $NPV=-13.06$ \linebreak
 
 We should undertake investment A because of its positive NPV and not investment B because of its negative NPV.
 
 \begin{problem}{6}

\end{problem}
 For the first option, we have 
 
 $PV=\frac{1,200}{1.065}+\frac{1,200}{1.065^2}+\frac{1,200}{1.065^3}+\frac{1,200}{1.065^4}$ hence $PV=\$4,110.96$
 
 For the second option to be equivalent, they have to have to same PV. Let p be the maximum price to pay in the final payment.
 
 $p=PV \times 1.065^4$ and we have $p=\$5,288.6$
 
 \begin{problem}{7}

\end{problem}
 
 For the first question we have $p_1$  the amount of money in the account at your $35^{th}$ birthday,  $p_1=21,000 \times 1.05^{17}$, if left untouched, there would be $p_1=\$48,132.38$\linebreak
 
 The amount of money originally put in the account $p_2$ is $p_2=\frac{21,000}{1.05^{18}}$. We have $p_2=\$8,725.93$
 
 
 \begin{problem}{8}

\end{problem}
 
The growth rate is equal the interest rate, therefore, the amount of money p that we have to have in bank is just $p=15,000\times12 \Rightarrow p=\$180,000$
 
 
  \begin{problem}{9}

\end{problem}

For a growing perpetuity, we know that $PV=\frac{C}{r+g}$ with $C=\$2000$, $r=0.02$ and $g=0.04 \Rightarrow
PV=\$3,333.3$


 \begin{problem}{10}

For the first question, assuming the repayment of the mortgage begins at year 1, the annuity will be $C=\frac{\$400,000 \times 0.09}{1-\frac{1}{1.09^{25}}} \Rightarrow C=\$40,7223.5$  \newline


For the second question, the present value payed after 25 years is \linebreak
$PV=\frac{38,000}{0.09} \times(1-\frac{1}{1.09^{30}}) \Rightarrow PV=\$390,398 $ 

The future value 30 years later is $FV=PV\times 1.09^{30} \Rightarrow FV=\$5,179,675$\newline
The future value of the loan is $FV(loan)=400,000\times 1.09^{30}\Rightarrow FV(loan)=\$5,307,707$\linebreak
With p the balloon payment we have  $p=FV(loan)-FV\Rightarrow p=\$128,032$
\end{problem}

 \begin{problem}{11}

\end{problem}

We want to have a future value at 66 of FV=2,500,000\$, so we have $PV=\frac{FV}{1.05^{39}} \Rightarrow PV=\$653,531$

We want to know the first transfer C to put in the saving accounts such as\linebreak $C=\frac{PV}{\frac{1}{r-g}\times(1-\frac{1+g}{1+r}^{39})}$ with r=0.035 and g=0.025. \newline
We obtain $C=\$20,733$


 \begin{problem}{12}

\end{problem}
 
 We call the Net Present Value after n year $NPV_n$ and we have \newline 
 $NPV_n=-250,000+\frac{30,000}{1.06} \times \frac{1-\frac{1}{1.06^n}}{1-\frac{1}{1.06}}$
 
 Solving to know when this value becomes 0 so that grandmother comes ahead we have 
 
 $1-\frac{1}{1.06^n}=\frac{1}{2}\Rightarrow n=11.89$
 
 Therefore, grandmother would have to live at leat 12 years to come ahead. 
 
\end{document}